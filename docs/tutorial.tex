\documentclass[fontset=windows]{ctexart}
\usepackage{hyperref}
\hypersetup{
    colorlinks=true,
    linkcolor=blue,
    filecolor=magenta,      
    urlcolor=cyan,
    pdftitle={Overleaf Example},
    pdfpagemode=FullScreen,
}
\title{水下SLAM仿真系统文档}
\author{肖书奇\footnote{邮箱xiaosq2000@gmail.com;微信号xiaosq2000}}
\date{}
\begin{document}
\maketitle
\begin{enumerate}
    \item 操作系统及中间件:Ubuntu 20.04 LTS, ROS Noetic
    \item 水下机器人仿真器:UUV Simulator
    \item 多波束声呐仿真器:NPS Multibeam Simulator (需要NVIDIA CUDA)
    \item 测试用例包:demo description
\end{enumerate}
\section{水下机器人仿真器}
仔细阅读\href{https://uuvsimulator.github.io/packages/uuv_simulator/intro/}{ UUV Simulator 项目主页},内有项目简介,功能介绍,开源代码仓库链接,使用教程等。
\par 尽管该项目未有 ROS Noetic 的官方发布,但经过本人测试,自行编译源代码可在Noetic环境下正常工作,其 \href{https://github.com/uuvsimulator/uuv_simulator/wiki}{Github Wiki} 介绍了 kinetic 版本的手动编译方法,注意不要机械地复制指令并执行,请将 kinetic 替换为 noetic,替换Gazebo版本号,替换你的catkin工作空间的路径等。另外,该Wiki中含有基本示例,建议实操体验。
\section{多波束声呐仿真器}
仔细阅读\href{https://field-robotics-lab.github.io/dave.doc/contents/dave_sensors/Multibeam-Forward-Looking-Sonar/}{ NPS Multibeam Sonar 项目主页},该项目的文档非常丰富(本系统不需要DAVE Project的其他组件,但建议充分学习)。
\par 需要注意的是,在安装 CUDA 之前,请确保你的 NVIDIA 显卡驱动正常安装,如何在Ubuntu系统安装NVIDIA驱动,以及如何检测,请参阅其他网上资源。
\par 另外,其依赖 \href{https://github.com/apl-ocean-engineering/hydrographic_msgs}{hydrographic\_msgs} 仍在不断更新,截至今日其main分支已经与nps\_multibeam\_simulator不兼容,故请git checkout到合适的历史节点,比如9fcdbe61bc781368b08e43a3d1af632b0ce08bc1。
\section{测试用例包}
针对实验室环境的测试用例包主要包含仿真声呐与Rexrov2的集成,基于xarco的水池场景模型(比起SDF文件更简洁,更方便调整尺寸),UUV\_Simulator 水下仿真环境中Oculus M1200D仿真的参数调整三项工作,该ROS软件包的结构并不复杂,主要涉及xarco格式的模型参数文件以及roslaunch文件,可在ROS与Gazebo中的官方文档中学习。将该软件包其置于你的工作空间,正确配置路径即可使用。
\end{document}